% Format teze zasnovan je na paketu memoir
% http://tug.ctan.org/macros/latex/contrib/memoir/memman.pdf ili
% http://texdoc.net/texmf-dist/doc/latex/memoir/memman.pdf
% 
% Prilikom zadavanja klase memoir, navedenim opcijama se podešava 
% veličina slova (12pt) i jednostrano štampanje (oneside).
% Ove parametre možete menjati samo ako pravite nezvanične verzije
% mastera za privatnu upotrebu (na primer, u b5 varijanti ima smisla 
% smanjiti 
\documentclass[12pt,oneside]{memoir}

% Paket koji definiše sve specifičnosti mastera Matematičkog fakulteta
\usepackage{assets/matfmaster}
%
% Podrazumevano pismo je ćirilica.
%   Ako koristite pdflatex, a ne xetex, sav latinički tekst na srpskom jeziku
%   treba biti okružen sa \lat{...} ili \begin{latinica}...\end{latinica}.
%
% Opicija [latinica]:
%   ako želite da pišete latiniciom, dodajte opciju "latinica" tj.
%   prethodni paket uključite pomoću: \usepackage[latinica]{matfmaster}.
%   Ako koristite pdflatex, a ne xetex, sav ćirilički tekst treba biti
%   okružen sa \cir{...} ili \begin{cirilica}...\end{cirilica}.
%
% Opcija [biblatex]:
%   ako želite da koristite reference na više jezika i umesto paketa
%   bibtex da koristite BibLaTeX/Biber, dodajte opciju "biblatex" tj.
%   prethodni paket uključite pomoću: \usepackage[biblatex]{matfmaster}
%
% Opcija [b5paper]:
%   ako želite da napravite verziju teze u manjem (b5) formatu, navedite
%   opciju "b5paper", tj. prethodni paket uključite pomoću: 
%   \usepackage[b5paper]{matfmaster}. Tada ima smisla razmisliti o promeni
%   veličine slova (izmenom opcije 12pt na 11pt u \documentclass{memoir}).
%
% Naravno, opcije je moguće kombinovati.
% Npr. \usepackage[b5paper,biblatex]{matfmaster}


% Paket koji obezbeđuje ispravni prikaz ćiriličkih italik slova kada
% se koristi pdflatex. Zakomentarisati ako na sistemu koji koristite ovaj
% paket nije dostupan ili ako ne radi ispravno.
\usepackage{cmsrb}

% Ostali paketi koji se koriste u dokumentu
\usepackage{listings} % listing programskog koda

% Datoteka sa literaturom u BibTex tj. BibLaTeX/Biber formatu
\bib{master_thesis}

% Ime kandidata na srpskom jeziku (u odabranom pismu)
\autor{Ивона Милутиновић}
% Naslov teze na srpskom jeziku (u odabranom pismu)
\naslov{Развој апликације за асистенцију вежбачима ослањањем на корисничке податке употребом Аndroid и Django оквира}
% Godina u kojoj je teza predana komisiji
\godina{2024}
% Ime i afilijacija mentora (u odabranom pismu)
% TODO: Add mentor and commission
\mentor{др Иван \textsc{Чукић}, доцент \\
Универзитет у Београду, Математички факултет}
%\mentor{др Мика \textsc{Микић}, редован професор\\ Универзитет у Београду, Математички факултет}
\komisijaA{проф. др Саша \textsc{Малков}, ванредни професор \\
Универзитет у Београду, Математички факултет}
\komisijaB{др Богдан \textsc{Павковић}, доцент \\
Универзитет у Новом Саду, Факултет техничких наука}
% Ime i afilijacija prvog člana komisije (u odabranom pismu)
% \komisijaA{др Ана \textsc{Анић}, ванредни професор\\ University of Disneyland, Недођија}
% Ime i afilijacija drugog člana komisije (u odabranom pismu)
% \komisijaB{др Лаза \textsc{Лазић}, доцент\\ Универзитет у Београду, Математички факултет}
% Ime i afilijacija trećeg člana komisije (opciono)
% \komisijaC{}
% Ime i afilijacija četvrtog člana komisije (opciono)
% \komisijaD{}
% Datum odbrane (obrisati ili iskomentarisati narednu liniju ako datum odbrane nije poznat)
\datumodbrane{дан. месец 2024.}

% Apstrakt na srpskom jeziku (u odabranom pismu)
\apstr{%
TODO
}

% Ključne reči na srpskom jeziku (u odabranom pismu)
\kljucnereci{машинско учење, рекурентне неуронске мреже, Android, Django REST Framework, scrapy, план тренинга, трчање}

\begin{document}
% ==============================================================================
% Uvodni deo teze
\frontmatter
% ==============================================================================
% Naslovna strana
\naslovna
% Strana sa podacima o mentoru i članovima komisije
\komisija
% Strana sa posvetom (u odabranom pismu)
\posveta{}
% Strana sa podacima o disertaciji na srpskom jeziku
\apstrakt
% Sadržaj teze
\tableofcontents*

% ==============================================================================
% Glavni deo teze
\mainmatter
% ==============================================================================

% ------------------------------------------------------------------------------
\chapter{Увод}
ТОДО: Описати шта је одрађено у којој глави
% ------------------------------------------------------------------------------

\section{TODO}



% ------------------------------------------------------------------------------

\chapter{Android}


\chapter{Django и Django REST Framework}


\chapter{Неуронске мреже}


\chapter{Имплементација апликације Train wiser}
\section{Преглед коришћених технологија}
OAuth 2.0 је протокол за ауторизацију који омогућава веб сајту или апликацији треће стране да добије ограничен приступ другом HTTP сервису у име власника ресурса. Овај протокол обезбеђује безбедан приступ корисничким подацима од стране других сервиса без прослеђивања креденцијала за пријављивање апликацији треће стране осигуравајући повећану сигурност и олакшано корисничко искуство.
\section{Архитектура и дизајн апликације}
\section{Могућа унапређења}

\chapter{Коришћење апликације Train wiser}


\chapter{Имплементација модела машинског учења}
\section{Прикупљање података}
\section{Евалуација модела}
\section{Могућа унапређења}

% ------------------------------------------------------------------------------


% ------------------------------------------------------------------------------
\chapter{Закључак}
ТОДО: Описати укратко шта је одрађено у раду и шта је закључак
% ------------------------------------------------------------------------------


% ------------------------------------------------------------------------------
% Literatura
% ------------------------------------------------------------------------------
\literatura
\nocite{*}


% ==============================================================================
% Završni deo teze i prilozi
\backmatter
% ==============================================================================

% ------------------------------------------------------------------------------
% Biografija kandidata
\begin{biografija}
\end{biografija}
% ------------------------------------------------------------------------------

\end{document} 